%%% LaTeX Template: Two column article
%%%
%%% Source: http://www.howtotex.com/
%%% Feel free to distribute this template, but please keep to referal to http://www.howtotex.com/ here.
%%% Date: February 2011

%%% Preamble
\documentclass[	DIV=calc,%
							paper=a4,%
							fontsize=11pt,%
							twocolumn]{scrartcl}	 					% KOMA-article class

\usepackage{lipsum}													% Package to create dummy text

\usepackage[english]{babel}										% English language/hyphenation
\usepackage[protrusion=true,expansion=true]{microtype}				% Better typography
\usepackage{amsmath,amsfonts,amsthm}					% Math packages
\usepackage[pdftex]{graphicx}									% Enable pdflatex
\usepackage[svgnames]{xcolor}									% Enabling colors by their 'svgnames'
\usepackage[hang, small,labelfont=bf,up,textfont=it,up]{caption}	% Custom captions under/above floats
\usepackage{epstopdf}												% Converts .eps to .pdf
\usepackage{subfig}													% Subfigures
\usepackage{booktabs}												% Nicer tables
\usepackage{fix-cm}													% Custom fontsizes
\usepackage[utf8]{inputenc}


%%% Custom sectioning (sectsty package)
\usepackage{sectsty}													% Custom sectioning (see below)
\allsectionsfont{%															% Change font of al section commands
	\usefont{OT1}{phv}{b}{n}%										% bch-b-n: CharterBT-Bold font
	}

\sectionfont{%																% Change font of \section command
	\usefont{OT1}{phv}{b}{n}%										% bch-b-n: CharterBT-Bold font
	}



%%% Headers and footers
\usepackage{fancyhdr}												% Needed to define custom headers/footers
	\pagestyle{fancy}														% Enabling the custom headers/footers
\usepackage{lastpage}	

% Header (empty)
\lhead{}
\chead{}
\rhead{}
% Footer (you may change this to your own needs)
\lfoot{\footnotesize \texttt{Thibaut EHLINGER} \textbullet ~Comparaison des algorithmes TCP}
\cfoot{}
\rfoot{\footnotesize page \thepage\ of \pageref{LastPage}}	% "Page 1 of 2"
\renewcommand{\headrulewidth}{0.0pt}
\renewcommand{\footrulewidth}{0.4pt}



%%% Creating an initial of the very first character of the content
\usepackage{lettrine}
\newcommand{\initial}[1]{%
     \lettrine[lines=3,lhang=0.3,nindent=0em]{
     				\color{DarkGoldenrod}
     				{\textsf{#1}}}{}}



%%% Title, author and date metadata
\usepackage{titling}															% For custom titles

\newcommand{\HorRule}{\color{DarkGoldenrod}%			% Creating a horizontal rule
									  	\rule{\linewidth}{1pt}%
										}
										
										\renewcommand\thesubsubsection{}

\pretitle{\vspace{-30pt} \begin{flushleft} \HorRule 
				\fontsize{40}{40} \usefont{OT1}{phv}{b}{n} \color{DarkRed} \selectfont 
				}
\title{Comparaison des principaux algorithmes TCP}					% Title of your article goes here
\posttitle{\par\end{flushleft}\vskip 0.5em}

\preauthor{\begin{flushleft}
					\large \lineskip 0.5em \usefont{OT1}{phv}{b}{sl} \color{DarkRed}}
\author{Thibaut Ehlinger, }											% Author name goes here
\postauthor{\footnotesize \usefont{OT1}{phv}{m}{sl} \color{Black} 
					Université de Strasbourg 								% Institution of author
					\par\end{flushleft}\HorRule}

\date{20 octobre 2015}																			



%%% Begin document
\begin{document}
\maketitle
\thispagestyle{fancy} 			% Enabling the custom headers/footers for the first page 
% The first character should be within \initial{}
\initial{T}\textbf{CP (\textit{Transmission Control Protocol})} est une implémentation de la couche transport du modèle OSI. C'est de loin son implémentation la plus populaire, avec pour maigre concurrent UDP, qui est moins contraignant et donc plus utile pour les connexions ne nécéssitant pas une grand robustesse. TCP étant omniprésent dans les réseaux actuels, il a donc souvent été étudié et réimplémenté, parfois parce qu'une implémentation présentait des faiblesses qui lui étaient inhérentes, parfois parce qu'aucune implémentation ne répondait à une problématique bien précise du réseau l'utilisant, comme c'est le cas pour les \textit{long fat network}, par exemple. Dans cet rédaction, nous étudierons tout d'abord les différentes implémentations historiques de TCP. Celles ayant été utilisées un jour ou l'autre mais qui sont ajourd'hui obsolètes car peu efficaces. Puis nous étudierons certains des algorithmes utilisés aujourd'hui, dans le but de comprendre quels sont les critères déterminants lors du choix de l'algorithme utilisé. En effet ceux-ci peuvent-être plus ou moins "aggressifs" ou stables.

\section*{Les premières implémentations de TCP}
De TCP Tahoe à TCP Westwood, en passant par TCP Reno, nous allons vous présenter une série d'algorithmes qui se sont globalement succédés, chacun étant dans la plupart des cas plus efficace que son prédecesseur, en tout point.

\subsection*{TCP Tahoe}
\subsubsection*{Slow start \& congestion avoidance}


Cette implémentation est citée pour la première fois en 1988 dans un article de Van Jacobson et Karels. C'est la plus simple et la moins efficace, tout types de problème confondus. Au début, on est en \textbf{Slow Start} : tant qu'on n'a aucune perte, on double la taille de la \textit{congestion window} (\textbf{CWND}), c'est à dire la taille en segments des rafales TCP envoyées. À chaque fois que tous les segments envoyés sont acquittés, on double la taille de la CWND. Le slow start est interrompu dès la première perte de paquet, c'est à dire lorsqu'on reçoit plusieurs fois le même acquittement (\textbf{ACK}) ou lors d'un \textbf{time out}. 

Comme la plupart des pertes de paquets sur les réseaux filaires sont provoqués par des congestions, Tahoe implémente le mode \textbf{Congestion Avoidance (CA)}. On se base sur une variable nommée \textbf{sstresh} (\textit{slow-start threshold}). Lorsque $CWND <= sstresh$, on est en Slow Start, sinon on est en CA. En CA, on augmente la taille de la fenêtre de 1 segment à chaque fois que tous les segments d'une rafale sont acquittés. C'est donc une progression linéaire.

\subsubsection*{Fast retransmit}

Si un ACK est reçu trois fois de suite, alors le segment (ACK+1) a probablement été perdu. Il ne s'agit probablement pas d'un déséquencement. Pour pallier ce problème, Tahoe renvoie directement le segment concerné au lieu d'attendre la fin d'un timeout. Après un fast retransmit, $sstresh=\frac{CWND}{2} $ et $ CWND=1$ (Slow Start).

\subsubsection*{Avantages et faiblesses}
En résumé, Tahoe est déjà une réelle avancée, car il permet d'approximer très grossièrement la plus grosse taille de fenêtre d'émission possible, et il reste stable à moins qu'il n'y aie des congestions sur le réseau. Dans ce cas, Tahoe peut s'adapter au problème et diminuer la taille de sa fenêtre d'émission.

Néanmoins, la réinitialisation de la fenêtre d'émission à \textbf{1} à la moindre perte de paquet est une sous-estimation beaucoup trop importante. La taille moyenne de CWND est donc bien plus faible que ce qu'elle pourrait être optimalement.


\subsection*{TCP Reno}
TCP résoud déjà grandement le problème de l'oscillation perpetuelle Slow Start/CA en ajoutant la notion de \textbf{Fast Recovery}.

\subsubsection*{Fast recovery}
Après un fast retransmit, au lieu de passer en Slow Start on applique :
$ sstresh=\frac{CWND}{2}, CWND=sstresh+3$. En fait on repasse directement en CA, tout en augmentant la taille de CWND de 3, en référence aux trois segments qui n'ont pas été reçus à cause des ACK dupliqués. En cas de time out néanmoins, on repasse en Slow Start. Cette stratégie permet en cas de perte de ne pas baisser drastiquement la taille de CWND, à moins d'un time out. Un time out est en fait plus grave qu'une simple perte de paquet. Il témoigne probablement d'une modification topologique du réseau ou bien d'une congestion importante, alors qu'une simple perte de paquet peut-être due à des évènements plus ponctuels, tels qu'un déséquencement ou une perte.

\subsubsection*{Faiblesses}

TCP Reno réagit bien au pertes de paquets quand elles se limitent à une par rafale. Quand par contre il y en a plusieurs par rafale, Reno est presque aussi inefficace que Tahoe, car il ne peut détecter qu'une seule perte de paquet à la fois.

Autre problème : dans certains cas, CWND peut être diminué deux fois en cas de pertes multiples dans une même rafale.
\section*{Heading on level 1 again}
Lorem ipsum dolor sit amet, consectetuer adipiscing elit. Aenean commodo ligula eget dolor. Aenean massa. Cum sociis natoque penatibus et magnis dis parturient montes, nascetur ridiculus mus. Donec quam felis, ultricies nec, pellentesque eu, pretium quis, sem. Nulla consequat massa quis enim. Donec pede justo, fringilla vel, aliquet nec, vulputate eget, arcu. In enim justo, rhoncus ut, imperdiet a, venenatis vitae, justo. Nullam dictum felis eu pede mollis pretium. Integer tincidunt. Cras dapibus. Vivamus elementum semper nisi. Aenean vulputate eleifend tellus. Aenean leo ligula, porttitor eu, consequat vitae, eleifend ac, enim. Aliquam lorem ante, dapibus in, viverra quis, feugiat a, tellus. Phasellus viverra nulla ut metus varius laoreet. Quisque rutrum. Aenean imperdiet. Etiam ultricies nisi vel augue. Curabitur ullamcorper ultricies 

\begin{table}
\caption{Random table}
\centering
	\begin{tabular}{llr}
		\toprule
		\multicolumn{2}{c}{Name} \\
		\cmidrule(r){1-2}
			First name & Last Name & Grade \\
		\midrule
			John & Doe & $7.5$ \\
			Richard & Miles & $2$ \\
		\bottomrule
	\end{tabular}
\end{table}

\subsection*{Heading on level 2}
Lorem ipsum dolor sit amet, consectetuer adipiscing elit. Aenean commodo ligula eget dolor. Aenean massa. Cum sociis natoque penatibus et magnis dis parturient montes, nascetur ridiculus mus. Donec quam felis, ultricies nec, pellentesque eu, pretium quis, sem. 

Lorem ipsum dolor sit amet, consectetuer adipiscing elit. Aenean commodo ligula eget dolor. Aenean massa. Cum sociis natoque penatibus et magnis dis parturient montes, nascetur ridiculus mus. Donec quam felis, ultricies nec, pellentesque eu, pretium quis, sem. Nulla consequat massa quis enim. Donec pede justo, fringilla vel, aliquet nec, vulputate eget, arcu. In enim justo, rhoncus ut, imperdiet a, venenatis vitae, justo. 
\begin{description}
	\item[First] This is the first item
	\item[Last] This is the last item
\end{description}
Nullam dictum felis eu pede mollis pretium. Integer tincidunt. Cras dapibus. Vivamus elementum semper nisi. Aenean vulputate eleifend tellus. Aenean leo ligula, porttitor eu, consequat vitae, eleifend ac, enim. Aliquam lorem ante, dapibus in, viverra quis, feugiat a, tellus. Phasellus viverra nulla ut metus varius laoreet. Quisque rutrum. Aenean imperdiet. Etiam ultricies nisi vel augue. Curabitur ullamcorper ultricies 

\end{document}